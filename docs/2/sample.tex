\documentclass[a4j,12pt]{jarticle}

\usepackage [dvipdfmx]{graphicx}
\usepackage {amsmath}

\title {pLaTeXシステムの利用}
\author {脇田 建 \and 米崎 直樹}
\date {}

\begin {document}

\maketitle

\begin {abstract}
文書組版システムpLaTeXの使い方を解説します。pLaTeXを利用するときには、文章はその内容のなかに組版用のコマンドを交ぜ書きします。このファイルをpLaTeXシステムで処理すると、市販の書物と同等の美しい文書が作れます。
\end {abstract}

\section {pLaTeXシステムとは}

スタンフォード大学のクヌース教授が数学の論文を美しく印刷する目的でTeXシステムを開発しました。TeXは複雑な数式を含んだ科学技術系の文書を生成し、その美しさは出版社も顔負けの品質でした。クヌース教授はTeXシステムを一般の人々が利用することは想定していませんでした。このため、TeXシステムをそのまま利用することには大きな苦痛が伴いました。のちにLamport博士がTeXシステムを使いやすく拡張したLaTeXシステムを発表すると、LaTeXシステムは科学技術分野の研究者に広く受け入れられました。

もともとのTeXシステムは欧米の言語での利用が想定されていましたが、日本にLaTeXを日本語化するコミュニティが生れ、やがてpLaTeXシステムが生まれました。pLaTeXは縦書きの文書やUnicodeに含まれる古い時代の文字も扱うことができるため、科学技術分野だけでなく文科系の研究者にも利用されるようになりました。

\end {document}
